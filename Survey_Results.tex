% Options for packages loaded elsewhere
\PassOptionsToPackage{unicode}{hyperref}
\PassOptionsToPackage{hyphens}{url}
%
\documentclass[
]{article}
\usepackage{amsmath,amssymb}
\usepackage{iftex}
\ifPDFTeX
  \usepackage[T1]{fontenc}
  \usepackage[utf8]{inputenc}
  \usepackage{textcomp} % provide euro and other symbols
\else % if luatex or xetex
  \usepackage{unicode-math} % this also loads fontspec
  \defaultfontfeatures{Scale=MatchLowercase}
  \defaultfontfeatures[\rmfamily]{Ligatures=TeX,Scale=1}
\fi
\usepackage{lmodern}
\ifPDFTeX\else
  % xetex/luatex font selection
\fi
% Use upquote if available, for straight quotes in verbatim environments
\IfFileExists{upquote.sty}{\usepackage{upquote}}{}
\IfFileExists{microtype.sty}{% use microtype if available
  \usepackage[]{microtype}
  \UseMicrotypeSet[protrusion]{basicmath} % disable protrusion for tt fonts
}{}
\makeatletter
\@ifundefined{KOMAClassName}{% if non-KOMA class
  \IfFileExists{parskip.sty}{%
    \usepackage{parskip}
  }{% else
    \setlength{\parindent}{0pt}
    \setlength{\parskip}{6pt plus 2pt minus 1pt}}
}{% if KOMA class
  \KOMAoptions{parskip=half}}
\makeatother
\usepackage{xcolor}
\usepackage[margin=1in]{geometry}
\usepackage{graphicx}
\makeatletter
\def\maxwidth{\ifdim\Gin@nat@width>\linewidth\linewidth\else\Gin@nat@width\fi}
\def\maxheight{\ifdim\Gin@nat@height>\textheight\textheight\else\Gin@nat@height\fi}
\makeatother
% Scale images if necessary, so that they will not overflow the page
% margins by default, and it is still possible to overwrite the defaults
% using explicit options in \includegraphics[width, height, ...]{}
\setkeys{Gin}{width=\maxwidth,height=\maxheight,keepaspectratio}
% Set default figure placement to htbp
\makeatletter
\def\fps@figure{htbp}
\makeatother
\setlength{\emergencystretch}{3em} % prevent overfull lines
\providecommand{\tightlist}{%
  \setlength{\itemsep}{0pt}\setlength{\parskip}{0pt}}
\setcounter{secnumdepth}{-\maxdimen} % remove section numbering
\newlength{\cslhangindent}
\setlength{\cslhangindent}{1.5em}
\newlength{\csllabelwidth}
\setlength{\csllabelwidth}{3em}
\newlength{\cslentryspacingunit} % times entry-spacing
\setlength{\cslentryspacingunit}{\parskip}
\newenvironment{CSLReferences}[2] % #1 hanging-ident, #2 entry spacing
 {% don't indent paragraphs
  \setlength{\parindent}{0pt}
  % turn on hanging indent if param 1 is 1
  \ifodd #1
  \let\oldpar\par
  \def\par{\hangindent=\cslhangindent\oldpar}
  \fi
  % set entry spacing
  \setlength{\parskip}{#2\cslentryspacingunit}
 }%
 {}
\usepackage{calc}
\newcommand{\CSLBlock}[1]{#1\hfill\break}
\newcommand{\CSLLeftMargin}[1]{\parbox[t]{\csllabelwidth}{#1}}
\newcommand{\CSLRightInline}[1]{\parbox[t]{\linewidth - \csllabelwidth}{#1}\break}
\newcommand{\CSLIndent}[1]{\hspace{\cslhangindent}#1}
\usepackage{booktabs}
\usepackage{longtable}
\usepackage{array}
\usepackage{multirow}
\usepackage{wrapfig}
\usepackage{float}
\usepackage{colortbl}
\usepackage{pdflscape}
\usepackage{tabu}
\usepackage{threeparttable}
\usepackage{threeparttablex}
\usepackage[normalem]{ulem}
\usepackage{makecell}
\usepackage{xcolor}
\ifLuaTeX
  \usepackage{selnolig}  % disable illegal ligatures
\fi
\IfFileExists{bookmark.sty}{\usepackage{bookmark}}{\usepackage{hyperref}}
\IfFileExists{xurl.sty}{\usepackage{xurl}}{} % add URL line breaks if available
\urlstyle{same}
\hypersetup{
  pdftitle={Survey Results},
  hidelinks,
  pdfcreator={LaTeX via pandoc}}

\title{Survey Results}
\author{}
\date{\vspace{-2.5em}}

\begin{document}
\maketitle

There were ten dyads enrolled in this study over the course of three
recruitment waves. A dyad consisted of a participant with Parkinson's
Disease and their caregiver (generally their spouse). In one dyad, there
was not a participating caregiver. One dyad participated in pretesting
but dropped the study before posttesting and is not included in this
analysis.

\begin{table}

\caption{\label{tab:DemographicsTableOutput}Demographics}
\centering
\begin{tabular}[t]{lllll}
\toprule
 &  & var & meansd & mediqr\\
\midrule
\addlinespace[0.3em]
\multicolumn{5}{l}{\textbf{Care Partner}}\\
\addlinespace[0.3em]
\multicolumn{5}{l}{\textit{Gender \(n, \%)}}\\
\hspace{1em}\hspace{1em} &  & Female & 5 (62.5\% ) & \\
\cmidrule{3-5}
\hspace{1em}\hspace{1em} &  & Male & 3 (37.5\% ) & \\
\cmidrule{2-5}
\addlinespace[0.3em]
\multicolumn{5}{l}{\textit{Age \(years)}}\\
\hspace{1em}\hspace{1em} &  &  & 68.5 (7.39) & 69 (12)\\
\cmidrule{2-2}
\cmidrule{4-5}
\addlinespace[0.3em]
\multicolumn{5}{l}{\textit{Education \(years)}}\\
\hspace{1em}\hspace{1em} &  &  & 14.57 (2.51) & 14 (3)\\
\cmidrule{1-5}
\addlinespace[0.3em]
\multicolumn{5}{l}{\textbf{PD Patient}}\\
\addlinespace[0.3em]
\multicolumn{5}{l}{\textit{Gender \(n, \%)}}\\
\hspace{1em}\hspace{1em} &  & Female & 4 (44.44\% ) & \\
\cmidrule{3-5}
\hspace{1em}\hspace{1em} &  & Male & 5 (55.56\% ) & \\
\cmidrule{2-5}
\addlinespace[0.3em]
\multicolumn{5}{l}{\textit{Age \(years)}}\\
\hspace{1em}\hspace{1em} &  &  & 70.22 (5.8) & 71 (8)\\
\cmidrule{2-2}
\cmidrule{4-5}
\addlinespace[0.3em]
\multicolumn{5}{l}{\textit{Education \(years)}}\\
\hspace{1em}\hspace{1em} &  &  & 14.56 (1.94) & 14 (2)\\
\cmidrule{2-2}
\cmidrule{4-5}
\addlinespace[0.3em]
\multicolumn{5}{l}{\textit{Age of PD Onset}}\\
\hspace{1em}\hspace{1em} &  &  & 68.78 (5.91) & 69 (10)\\
\cmidrule{2-2}
\cmidrule{4-5}
\addlinespace[0.3em]
\multicolumn{5}{l}{\textit{Hoen \& Yahr Score}}\\
\hspace{1em}\hspace{1em} &  &  & 2.22 (0.36) & 2 (0.5)\\
\cmidrule{2-2}
\cmidrule{4-5}
\addlinespace[0.3em]
\multicolumn{5}{l}{\textit{Symptom Duration \(years)}}\\
\hspace{1em}\hspace{1em} &  &  & 1.54 (1) & 1.5 (1.58)\\
\cmidrule{2-2}
\cmidrule{4-5}
\addlinespace[0.3em]
\multicolumn{5}{l}{\textit{Carbidopa-Levidopa Dose \(mg/day)}}\\
\hspace{1em}\hspace{1em} &  &  & 163.89 (71.93) - 655.56 (287.71) & 150 (75) - 600 (300)\\
\bottomrule
\end{tabular}
\end{table}

\hypertarget{geriatric-depression-scale-short-form}{%
\subsection{Geriatric Depression Scale (Short
Form)}\label{geriatric-depression-scale-short-form}}

The
\href{https://hign.org/consultgeri/try-this-series/geriatric-depression-scale-gds\#:~:text=Scores\%20of\%200\%2D4\%20are,12\%2D15\%20indicate\%20severe\%20depression.}{Geriatric
Depression Scale Short Form} is a fifteen item questionnaire. Each item
has binary answer choices `yes' and `no'. Items 1, 5, 7, 11, and 13 are
scored 1 point for `no' and 0 points for `yes'. Items 2, 3, 4, 6, 8, 9,
10, 12, 14, and 15 are scored 1 point for `yes' and 0 points for `no'.
Points are totaled and the total is evaluated as follows: a score less
than 5 is normal, 5-8 indicates mild depression, 9-11 indicates moderate
depression, and a score greater than 12 indicates severe depression.

At Pre-Test, all eight care partners scored in the `normal' category. At
Post-Test, all eight care partners scores in the `normal' category. At
Pre-Test, six out of nine patients with Parkinson's Disease scored in
the `normal' category and two patients scored in the `mild depression'
category (see Table {[}scoreinterpretations-table{]}). One patient
scored in the `severe depression' category at Pre-Test (see Table
{[}scoreinterpretations-table{]}). At Post-Test, the patient who scored
`severe depression' at Pre-Test scored in the `normal' category at
Post-Test. No patients scored in the `severe depression' category at
Post-Test. All patients with Parkinson's Disease who scored in the
`normal' category at Pre-Test scored in the `normal' category at
Post-Test.

According to a 2019 study, the minimal clinically important difference
in the Geriatric Depression Scale (15 Items) for improvement and
deterioration were 5 and 4 points respectively (Quinten et al. 2019).
The largest change in score among care partners was a decrease of two
points. Three care partners had no change in score. One patient with
Parkinson's Disease had a decrease in score of eight points. This
patient scored in the `severe depression' category at Pre-Test and
scored in the `normal' category at Post-Test and was the only patient to
participate in the study without a care partner. None of the other eight
patients with Parkinson's Disease had a change in score greater than two
points. Using the minimal clinically important difference cutoffs from
the study reported above, one patient with Parkinson's Disease and no
care partners reached minimal clinically important difference in GDS15
scores.

We are 95 \(\%\) confident that the true change in Geriatric Depression
Scale scores for patients with Parkinson's Disease after a tandem
cycling intervention is between -3.33 and 0.66 points (t(8) = -1.54, p =
0.16).

We are 95 \(\%\) confident that the true change in Geriatric Depression
Scale scores for care partners after a tandem cycling intervention is
between -1.57 and -0.18 points (t(7) = -2.97, p = 0.02).

\hypertarget{general-anxiety-disorder---7}{%
\subsection{General Anxiety Disorder -
7}\label{general-anxiety-disorder---7}}

The
\href{https://www.hiv.uw.edu/page/mental-health-screening/gad-7\#:~:text=Score\%200\%2D4\%3A\%20Minimal\%20Anxiety,greater\%20than\%2015\%3A\%20Severe\%20Anxiety}{GAD-7}
is a seven item survey, where each item is a Lickert scale with choices
`Not at all', `Several Days', `More than half the days', and `Nearly
every day', scored as 0, 1, 2, and 3 respectively.

The seven items are totaled and the total score is interpreted as
follows: a score less than 5 indicates minimal anxiety, 5-9 indicates
mild anxiety, 10 - 14 indicated moderate anxiety, and 15 or greater
indicates severe anxiety.

At Pre-Test, six patients scored in the `minimal anxiety category', two
patients scored in the `mild anxiety' category, and one patient scored
in the `moderate anxiety' category. At Post-Test, five patients scored
in the `minimal anxiety' category and four patients scored in the `mild
anxiety category'. The patient who scored in the `moderate anxiety'
category at Pre-Test scored in the `minimal anxiety' category at
Post-Test. Three patients who scored in the `minimal anxiety' category
at Pre-Test scored in the `mild anxiety' category at Post-Test. One
patient who scored in the `mild anxiety' category at Pre-Test scored in
the `minimal anxiety' category at Post-Test.

At Pre-Test, seven care partners scored in the `minimal anxiety'
category and one care partner scored in the `mild anxiety' category. At
Post-Test, all eight care partners scored in the `mild anxiety'
category.

The minimal clinically important difference on the GAD-7 is four points
(Toussaint et al. 2020).

Using the minimal clininally important difference from the study cited
above, one care partner had a clinically important decrease in GAD7
score from Pre-Test to Post-Test. Two Parkinson's Disease patients had
clinically important decreases in GAD7 score from Pre-Test to Post-Test
and three Parkinson's Disease patients had a clinically important
increase in GAD7 score from Pre-Test to Post-Test.

We are 95 \(\%\) confident that the true change in General Anxiety
Disorder score for patients with Parkinson's Disease after a tandem
cycling intervention is between -3.88 and 4.32 points (t(8) = 0.12, p =
0.9).

We are 95 \(\%\) confident that the true change in General Anxiety
Disorder score for care partners after a tandem cycling intervention is
between -2.52 and 1.27 points (t(7) = -0.78, p = 0.46).

\hypertarget{brief-resilience-scale}{%
\subsection{Brief Resilience Scale}\label{brief-resilience-scale}}

The
\href{https://measure.whatworkswellbeing.org/measures-bank/brief-resilience-scale/}{Brief
Resilience Scale} is a six item survey with answers using a five point
Lickert-scale. Items 1, 3, and 5 are positively worded and Items 2, 4,
and 6 are negatively worded, therefore Items 2, 4, and 6 are scored by
reverse coding the answers. The overall score is determined by taking
the mean of the six items.

When interpreting the score, less than 3.00 indicates low resilience,
3.00-4.30 indicates normal resilience, and greater than 4.30 indicates
high resilience.

At Pre-Test, six of the Parkinson's Disease Patients scored in the
`normal resilience' category and three scored in the `high resilience
category'. At Post-Test, one patient scored as `low resilience', 6
patients scored as `normal resilience', and two patients scored as `high
resilience'. One patient who scored as `normal resilience' at Pre-Test
scored as `low resilience' at Post-Test, one patient who scored as
`normal resilience' at Pre-Test scored as `high resilience' at
Post-Test, and two patients who scored as `high resilience' at Pre-Test
scored as `normal resilience' at Post-Test.

At Pre-Test, six care partners scored in the `normal resilience'
category and two care partners scored in the `high resilience' category.
At Post-Test, five care partners scored in the `normal resilience'
category and three care partners scored in the `high resilience'
category. Two care partners who scored in the `normal resilience'
category at Pre-Test scored in the `high resilience' category at
Post-Test and one care partner who scored in the `high resilience'
category at Pre-Test scored in the `normal resilience' category at
Post-Test.

We are 95 \(\%\) confident that the change in Brief Resilience Scale
score for care partners after a tandem cycling intervention is between
-0.39 and 0.64 points (t(7) = 0.57, p = 0.58).

We are 95 \(\%\) confident that the change in Brief Resilience Scale
score for patients with Parkinson's Disease after a tandem cycling
intervention is between -0.45 and 0.38 points (t(8) = -0.21, p = 0.84).

\hypertarget{revised-dyadic-adjustment-scale}{%
\subsection{Revised Dyadic Adjustment
Scale}\label{revised-dyadic-adjustment-scale}}

One patient completed the study without a caregiver and thus did not
complete this scale.

\url{https://relationshipinstitute.com.au/uploads/resources/Revised-Dyadic-Adjustment-Scale-RDAS-1.pdf}

The
\href{https://scholarsarchive.byu.edu/cgi/viewcontent.cgi?article=5608\&context=facpub}{Revised
Dyadic Adjustment Scale} is a fourteen item survey that measures seven
dimensions of a couple's relationship among three greater categories.
Each item is a 5 or 6 point Lickert scale. Items are summed and the
total score can range from 0-69. Higher scores indicate greater
relationship satisfaction and lower scored indicate greater relationship
distress. The overall cutoff score is 48 with scores below 48 indicating
relationship/marital distress and scores above 48 indicating
non-distress.

The Consensus dimension consists of Items 1-6 and measures consensus on
matters of importance to marital functioning. The Satisfaction dimension
consists of Items 7-10 and measures dyadic satisfaction. The Cohesion
dimension consists of Items 11-14 and measures dyadic cohesion.

When reviewing the RDAS, one participant had abnormal responses at post
test. These responses indicated that the participant, who had answered
the survey at pre-test indicating a very happy marriage, was now
considering divorce almost every day and was extremely unhappy in the
marriage. These answers were only on the four reverse coded questions
(7-10) and the participant's partner did not indicate any issues in the
marriage at post test compared to pre test. After discussion, we decided
to look at the data both including and excluding this participant's
data. RDAS scores were not significantly different from pre test to post
test whether this participant was included or not. This participant was
the only participant that indicated distress in the relationship at post
testing.

When using the Revised Dyadic Adjustment scale for longitudinal testing
or for studies where the subjects are tested at least twice, the RDA is
designed to be able to be split in half and have one question for each
subscale for a total of seven questions. The survey can be administered
this way with one half for pre-testing and the second half for
post-testing, which allows for control over bias from repeat testing.
Future research using this scale may benefit from using this process.

\hypertarget{promis---29-profile-v2.0}{%
\subsection{PROMIS - 29 Profile v2.0}\label{promis---29-profile-v2.0}}

The
\href{https://www.healthmeasures.net/images/PROMIS/manuals/Scoring_Manuals_/PROMIS_Adult_Profile_Scoring_Manual.pdf}{PROMIS
Adult Profile} is a 29 item survey. A
\href{https://staging.healthmeasures.net/score-and-interpret/interpret-scores/promis/promis-score-cut-points}{score
of 50} is considered average for the United States general adult
population. A higher score indicates more of the concept being measured.
The PROMIS-29 has the following subcategories: Physical Function,
Anxiety, Depression, Fatigue, Sleep Disturbance, Ability to Participate
in Social Roles and Activities, and Pain Interference. The items are
scored on a four-point Lickert scale.

The threshold to evaluate within-group change or to make a between-group
comparison generally ranges between 2 and 6 T-score points.

We are 95 \(\%\) confident that the true change in PROMIS-29 Fatigue
dimension t-score for patients with Parkinson's Disease after a tandem
cycling intervention lies between -10.88 and -1.1 points (t(8) = -2.83,
p = 0.02).

We are 95 \(\%\) confident that the true change in PROMIS-29 Social
Participation dimension t-score for patients with Parkinson's Disease
after a tandem cycling intervention lies between 2.62 and 6.76 points
(t(8) = 5.21, p = 0).

\hypertarget{pdq-39}{%
\subsection{PDQ 39}\label{pdq-39}}

For each dimension of the PDQ-39, the answers are summed across the
items and divided by the total possible points on the items answered.
This is then multiplied by 100 to obtain the dimension summary index.

Each dimension is scored from 0 to 100, with lower scores reflecting
better quality of life, therefore if using change in score calculated as
Post-Test score minus Pre-Test score, a negative difference would
reflect a decrease in score and therefore an improvement.

The minimum clinically important difference of PDQ-39 summary indices
was 4.22 for worsening and -4.72 for improvement (Horváth et al. 2017).

We are 95 \(\%\) confident that the change in PDQ-39 Mobility dimension
for patients with Parkinson's Disease after a tandem cycling
intervention lies between -23.96 and -3.26 points (t(8) = -3.03, p =
0.02).

\hypertarget{moca}{%
\subsection{MoCA}\label{moca}}

The Montreal Cognitive Assessment (MoCA) was administered at pre-test
and post-test for all participants. The MoCA has a total score between 0
and 30, with a score of 26 or higher considered `normal cognitive
function'. A study of cognitive function after a subarachnoid hemorrhage
caused by cerebral aneurysm determined that a minimum clinically
important difference in MoCA score associated with a change of health in
general was 2 points (Wong et al. 2017).

\hypertarget{references}{%
\subsection*{References}\label{references}}
\addcontentsline{toc}{subsection}{References}

\hypertarget{refs}{}
\begin{CSLReferences}{1}{0}
\leavevmode\vadjust pre{\hypertarget{ref-horvath2017}{}}%
Horváth, Krisztina, Zsuzsanna Aschermann, Márton Kovács, Attila Makkos,
Márk Harmat, József Janszky, Sámuel Komoly, Kázmér Karádi, and Norbert
Kovács. 2017. {``{Changes in Quality of Life in Parkinson's Disease: How
Large Must They Be to Be Relevant?}''} \emph{Neuroepidemiology} 48
(1-2): 1--8. \url{https://doi.org/10.1159/000455863}.

\leavevmode\vadjust pre{\hypertarget{ref-quinten2019}{}}%
Quinten, C., C. Kenis, L. Decoster, P. R. Debruyne, I. De Groof, C.
Focan, F. Cornelis, et al. 2019. {``Determining Clinically Important
Differences in Health-Related Quality of Life in Older Patients with
Cancer Undergoing Chemotherapy or Surgery.''} \emph{Quality of Life
Research} 28: 663--76. \url{https://doi.org/10.1007/s11136-018-2062-6}.

\leavevmode\vadjust pre{\hypertarget{ref-toussaint2020}{}}%
Toussaint, Anne, Paul Hüsing, Antje Gumz, Katja Wingenfeld, Martin
Härter, Elisabeth Schramm, and Bernd Löwe. 2020. {``Sensitivity to
Change and Minimal Clinically Important Difference of the 7-Item
Generalized Anxiety Disorder Questionnaire (GAD-7).''} \emph{Journal of
Affective Disorders} 265: 395--401.
\url{https://doi.org/10.1016/j.jad.2020.01.032}.

\leavevmode\vadjust pre{\hypertarget{ref-wong2017}{}}%
Wong, George Kwok Chu, Jodhy Suk Ying Mak, Adrian Wong, Vera Zhi Yuan
Zheng, Wai Sang Poon, Jill Abrigo, and Vincent Chung Tong Mok. 2017.
{``Minimum Clinically Important Difference of Montreal Cognitive
Assessment in Aneurysmal Subarachnoid Hemorrhage Patients.''}
\emph{Journal of Clinical Neuroscience} 46: 41--44.
\url{https://doi.org/10.1016/j.jocn.2017.08.039}.

\end{CSLReferences}

\end{document}
